% ****** Start of file apssamp.tex ******
%
%   This file is part of the APS files in the REVTeX 4.2 distribution.
%   Version 4.2a of REVTeX, December 2014
%
%   Copyright (c) 2014 The American Physical Society.
%
%   See the REVTeX 4 README file for restrictions and more information.
%
% TeX'ing this file requires that you have AMS-LaTeX 2.0 installed
% as well as the rest of the prerequisites for REVTeX 4.2
%
% See the REVTeX 4 README file
% It also requires running BibTeX. The commands are as follows:
%
%  1)  latex apssamp.tex
%  2)  bibtex apssamp
%  3)  latex apssamp.tex
%  4)  latex apssamp.tex
%
\documentclass[%
 reprint,
%superscriptaddress,
%groupedaddress,
%unsortedaddress,
%runinaddress,
%frontmatterverbose,
%preprint,
%preprintnumbers,
%nofootinbib,
%nobibnotes,
%bibnotes,
 amsmath,amssymb,
 aps,
%pra,
%prb,
%rmp,
%prstab,
%prstper,
%floatfix,
]{revtex4-2}

\usepackage{graphicx}% Include figure files
\usepackage{dcolumn}% Align table columns on decimal point
\usepackage{bm}% bold math
%\usepackage{hyperref}% add hypertext capabilities
%\usepackage[mathlines]{lineno}% Enable numbering of text and display math
%\linenumbers\relax % Commence numbering lines

%\usepackage[showframe,%Uncomment any one of the following lines to test
%%scale=0.7, marginratio={1:1, 2:3}, ignoreall,% default settings
%%text={7in,10in},centering,
%%margin=1.5in,
%%total={6.5in,8.75in}, top=1.2in, left=0.9in, includefoot,
%%height=10in,a5paper,hmargin={3cm,0.8in},
%]{geometry}

\begin{document}

\preprint{APS/123-QED}

\title{Enforced Swift Equilibrium on Manifolds}% Force line breaks with \\
%% \thanks{A footnote to the article title}%

\author{Shi-Fan Chen}
%% \altaffiliation[Also at ]{Physics Department, XYZ University.}%Lines break automatically or can be forced with \\
\author{Adrianne Zhong}%
 \email{Equal Contribution}
\author{Adam Frim}%
 \email{Equal Contribution}
\author{Dibuyendu Mandal}
\author{Michael DeWeese}
\affiliation{%
 Physics Department at UC Berkeley, \\ Go Bears! \\ \url{adrizhong.com/files/double-oski.jpg}
}%

\date{\today}% It is always \today, today,
             %  but any date may be explicitly specified

\begin{abstract}
ABSTRACT TO BE WRITTEN
\begin{description}

\section{Introduction}

The story is a familiar one. We have a physical system with microstates $X$, each of which with potential $V(x; \lambda)$, whereby the $\lambda$'s are control parameters we can manipulate. The probability distribution of the microstates then follows the Boltzmann Distribution: %
\begin{equation} \label{equilibrium}
    \rho_\mathrm{eq}(x; \lambda) = \big(Z(\lambda)\big)^{-1} \exp(-\beta V(x; \lambda))
\end{equation}%

whereby $\beta = (k_B T)^{-1}$ is the inverse temperature (and perhaps yet another control parameter), and $Z$ is the Boltzmann factor. In general, if you change $\lambda(t)$ slowly enough, the probability distribution $\rho(x, \lambda(t))$ remains infinitesimally close to $\rho_\mathrm{eq}(x, \lambda(t))$ as specified by Equation \ref{equilibrium}. When you change $\lambda(t)$ faster than the relaxation timescale $\tau_\mathrm{relax}$ (system-dependent), this is not the case.
\\
The million dollar question: is it possible to apply an external driving force to your system to offset the non-equilibrium perturbation terms, caused by changing $\lambda(t)$ in a non-adiabatic manner?

This is the problem of Enforced Swift Equilibrium that we are to solve and generalize to arbitrary Riemannian manifolds.
\begin{equation*} \end{equation*}%
\section{Langevin Dynamics}

In particular, we will start with consider a particle undergoing Browning motion in the overdamped regime, whose dynamics are governed by the Langevin equation:
%
\begin{equation}
    \gamma \frac{d\vec{x}}{dt} = - \vec{\nabla} V(\vec{x}; \lambda) % + \sqrt{\frac{2\gamma}{\beta}}\vec{\xi}(t) + \vec{f}_{\mathrm{ext}}(t)
    + \vec{\eta}(t) + \vec{F}_{\mathrm{ext}}(t)
\end{equation} %
\\ \\
whereby $\vec{x}$ is the position of the particle, $\gamma$ is the viscosity, $V_\lambda$ is the potential acting upon the particle, $\eta$ is noise with autoc-orrelation $< \eta(t') \eta(t) > = (2\gamma/\beta) \delta(t' - t)$, and $\vec{F}_{\mathrm{ext}}(t)$ is the external force we act upon the particle.
\\ \\
Supposing for now that both $\lambda$ and $ \vec{F}_{\mathrm{ext}}$ are constants in time, the well-known solution of the Fokker-Planck equation would solve:
\begin{align} \label{FP}
    \partial_t \rho &= \vec{\nabla} \cdot \bigg[ \bigg(\frac{\vec{\nabla} V(\vec{x}; \lambda) - \vec{F}_\mathrm{ext}}{\gamma} + \frac{\vec{\nabla} }{\beta \gamma} \bigg) \rho \bigg] \\
    &= \frac{ \vec{\nabla} \cdot \big[ - \rho \vec{F}_\mathrm{ext} +  \rho \vec{\nabla} V(\vec{x}; \lambda) + \beta^{-1} \vec{\nabla}\rho \ \big] }{ \gamma }
\end{align} %

By setting $\vec{F}_\mathrm{ext}$ to zero, we recover the steady state solution, namely equilibrium distribution $\rho_\mathrm{eq}$ from Equation \ref{equilibrium}.

Now suppose that the control parameters $\lambda(t)$ is time dependent. Subsequently, so is now the potential $V$, i.e. $\partial_t V \neq 0$. As was phrased in the introduction, we want to solve for $\vec{F}_\mathrm{ext}(t)$ that offsets the non-adiabatic perturbations, namely, so that $\rho(x; \lambda(t)) = \rho_\mathrm{eq}(x; \lambda(t))$
\\ \\
Using now the relations:
\begin{align*}
   % \partial_t \rho_\mathrm{eq} &= -\dot{\lambda} \partial_{\lambda} \big( \ln(Z(\lambda))  + \beta V(x; \lambda) \big)  \rho_{\mathrm{eq}} \\
   \vec{\nabla} \rho_\mathrm{eq} &= -\beta \rho_\mathrm{eq} \vec{\nabla} V(x; \lambda)
\end{align*}
\\
we arrive to the relation by plugging in the Boltzmann distribution (Eq. \ref{equilibrium}) to the Fokker-Planck solution (Eq. \ref{FP}):
\begin{align}
    -\gamma \partial_t \rho_\mathrm{eq} &= \vec{\nabla} \cdot \big( \rho_\mathrm{eq} \ \vec{F}_\mathrm{ext}  - \rho_\mathrm{eq} \vec{\nabla} V(\vec{x}; \lambda) - \beta^{-1} \vec{\nabla} \rho_\mathrm{eq} \big) \\
    % \gamma \dot{\lambda} \partial_{\lambda} \big( \ln(Z(\lambda))  + \beta V(x; \lambda) \big) \rho_\mathrm{eq}
    &= \vec{\nabla} \cdot \big[  \rho_\mathrm{eq} \ \vec{F}_\mathrm{ext} - \rho_\mathrm{eq} \vec{\nabla} V(\vec{x}; \lambda) - \beta^{-1}( -\beta \rho_\mathrm{eq} \vec{\nabla} V(x; \lambda)) \big] \\
    &= \vec{\nabla} \cdot \big[ \rho_\mathrm{eq} \ \vec{F}_\mathrm{ext}\big]
\end{align}

Defining know:
\begin{equation}
    \vec{P} := \rho_\mathrm{eq} \ \vec{F}_\mathrm{ext}
\end{equation}
we now have:
\begin{equation} \label{eq:ESE}
    \nabla \cdot \vec{P} =  -\gamma \partial_t \rho_\mathrm{eq} %\gamma \dot{\lambda} \partial_{\lambda} \big( \ln[Z_{\lambda}]  + \beta V_{\lambda}(x) \big) \rho_\mathrm{eq}
\end{equation}

wherein we rewrote $Z(\lambda)$ and $V(x; \lambda)$ as $Z_\lambda$ and $V_{\lambda}(x)$ respectively as to compacti-fy things. Equation \ref{eq:ESE} is the governing equation of ESE. In previous works, it has been solved in one dimension [cite]. In this paper, we will generalize this result to (1) higher dimesional spaces, and (2) abritrary, non-Euclidean manifolds.

\begin{equation*} \end{equation*}%
\\
\section{SOLUTION VIA THE HODGE DECOMPOSITION}

Consider that our phase space is no longer $\mathbb{R}^{d}$, but rather an arbitrary compact, orientable Riemannian manifold $M$ with metric $g$. Re-writing now the vector $\vec{P}$ as a differential one-form $P = P_i dx^{i}$, we can rewrite equation \ref{eq:ESE} as:
\begin{equation} \label{eq:ESEdiff}
  d^{\dagger} P = \gamma \dot{\lambda} \partial_{\lambda} \big( \ln[Z_{\lambda}]  + \beta V_{\lambda}(x) \big) \rho_\mathrm{eq}
\end{equation}
whereby $d^{\dagger}$ is the Hodge-dual exterior derivative operator.


Because the right hand side of Eq. \ref{eq:ESEdiff} is equal to $-\gamma \partial_t \rho_\mathrm{eq}$, we have by the conservation of probability flow:
\begin{equation}
    \int_{M} -\gamma \frac{\partial}{\partial t} \rho_\mathrm{eq} \ dV = -\gamma \frac{\partial}{\partial t} \int_{M} \rho_\mathrm{eq} \ dV = 0
\end{equation}


The Hodge Decomposition states that, for any k-form $P_k$ on a compact, orientable Riemannian manifold $M$, there exists a unique decomposition:
\begin{equation} \label{eq:hodgeD}
    P_k = d A_{(k-1)} + d^{\dagger}B_{(k+1)} + C_k
\end{equation}

whereby $d$ and $d^{\dagger}$ are respectively the exterior derivative and its Hodge dual, $A_{k-1}$ and $B_{k+1}$ are $(k-1)$- and $(k+1)$-forms, and $C_k$ is a harmonic $k$-form (i.e. $\Delta \gamma = 0$, where $\Delta = d d^{\gamma} + d^{\gamma} d$, the generalized Laplacian operator).

We plug  our $k=1$-form $P$ into Eq. $\label{eq:hodgeD}$ (i.e. $P = \omega_{k=1}$, and apply $d^{\gamma}$ to both sides to get:
\begin{align*}
  d^{\dagger} P &= d^{\dagger}(d A + d^{\dagger}B + C) \\
  &= d^{\dagger} d A + d^{\dagger} C \\
  &= \Delta A + d^{\dagger} C
\end{align*}

wherein $A$ is a $0$-form scalar, and $\gamma$ is a harmonic one-form. To get the second, we use $d^{\dagger}d^{\dagger} = 0$, and to get the third, we use $d (d^{\dagger} d \alpha_{0}) = d(0)$, as the exterior derivative of a constant/scalar zero-form is 0.  We will see that a general solution exists upon setting $B, C = 0$. (why is this the case?), to finally get:

We are now left with:
\begin{equation} \label{eq:ESEManifold}
  \Delta A = -\gamma \partial \rho_{\mathrm{eq}}
\end{equation}
wherein $\Delta$ is the Generalized Laplacian operator $d^{\dagger}d + d d^{\dagger}$.

Quick note: because $B$ vanishes, we can easily add/subtract it, so in a way we may change gauges by setting $B \neq 0$, and therefore making the substitution $P \leftarrow P - d^{\dagger} B $. (Remember that $P = \rho_{\mathrm{eq}} F_{\mathrm{ext}}\ \ $!)

\begin{equation*} \end{equation*}%
\section{Examples}

\\
\subsection{Higher Dimensional Euclidean Space}

\\
\subsection{Rigid Rotor}
Lettuce now consider a rigid rotor, whose phase space is the two dimensional sphere $M = S^2$. We have
\begin{equation}
    \Delta A = \frac{1}{\sin \theta} \frac{\partial}{\partial \theta} \bigg(\sin \theta \frac{\partial A}{\partial \theta} \bigg) + \frac{\partial^2 A}{\partial \phi^2}
\end{equation}
There exists a very-good eigenbasis for this operator in $S^2$, namely the spherical harmonics $\{Y_{l}^{m}(\theta, \phi) \}$, which we use in solving Equation \ref{eq:ESEManifold}.

A simple example is to consider an electric dipole with dipole moment $p$ sitting in a time-varying electrical field pointed in the z-direction $\vec{E} = E(t) \hat{z}$, which corresponds to the potential:

\begin{equation}
    V(\theta, \phi, t) = -p E(t) \cos (\theta)
\end{equation}

Because there is no $\phi$ dependence, the spherical harmonics reduce to the Lengendre polynomials $P_{l}(\cos \theta)$. This is supposed to be a Big-O.

Given the higher-temperature regime where $\beta^{-1} = k_{B} T >> |p E(t)|$, we can approximate the Boltzmann distribution:
\begin{align*}
    \rho(\theta, \phi, t) &\approx \frac{\exp(\beta p E(t) cos\theta)}{Z} \\
    &= \frac{1 + (\beta p E(t) \cos\theta)}{4 \pi}
\end{align*}
whereby we use that for this particular particular potential, $Z = 4\pi + \mathbb{O}((\beta|p E(t)|)^2)$. that's right, this is supposed to be a big-O notation.

We plug now everything into for Equation \ref{eq:ESEManifold}:
\begin{align}
    \Delta A = -\frac{\gamma \beta p \dot{E}(t)}{4\pi} \cos \theta
\end{align}
and we identify the right hand side as with the $l = 1$ Lengendre polynomial, i.e. $P_{l = 1}(\cos \theta) = \cos \theta$. Given that $\Delta P_{l}(\cos \theta) = -l(l+1) P_{l}(\cos \theta)$, we immediately get:
\begin{align}
    A(t) &= \frac{\gamma \beta p \dot{E}(t)}{8\pi} \cos \theta
\end{align}
To invert to get back $\vec{F}_{\mathrm{ext}}(t)$, we remember that $\vec{P} =  \vec{\nabla} A$, and that $\vec{F}_{\mathrm{ext}} = \vec{P} / \rho_{\mathrm{eq}} $
\begin{align}
    \vec{F}_{\mathrm{ext}} &= \frac{\vec{\nabla} A}{\rho_{\mathrm{eq}}} \\
    &\approx \frac{\gamma \beta p \dot{E}(t)}{2} \sin (\theta) \ \hat{\theta}
\end{align}
which is enforceable by adding an external electric field $\Delta E_{\mathrm{ESE}}(t) = \gamma \beta \dot{E} / 2$.
\\ \\
Intriguingly, if the original time-changing potential were a sinusoid, for instance $E(t) = E_0 \cos (\omega t)$, the ESE perturbation correction, to first order, would be to modulate the original signal with an exactly out-of-phase contribution to the original field,  $\Delta E_{\mathrm{ESE}}(t) = -(\gamma \beta \omega / 2) E_0 \sin (\omega t) $.
\\ \\
The resulting electric field would then be $E_{\mathrm{net}}(t) = A E_0 cos(\omega t + \phi) $, where $A = \sqrt{1 + (\gamma \beta \omega / 2)^2 }$, and $\phi = \tan^{-1}(\gamma \beta \omega / 2)$ -- a phase-shifted sinusoid from the original electric field.
\\ \\
In other words, to recover the equilibrium probability distribution for a sinusoidal electric field, you would apply a net electric field of the same frequency, but just phase-shifted in time, and with a slightly larger amplitude. The larger the sinusoidal frequency, the larger the phase- and ampitude-shifts.

\begin{equation*} \end{equation*}%
\\
\subsection{Pair o' Pendula}

The last example we will pursue is that of a Pair o' pendula, see Figure (draw a sketch of a pair of pendula).

\begin{equation}
  V(\theta_{1}, \theta_{2}) = g_1 \sin\big(\theta_1) + g_2 \sin(\theta_2 - \phi(t)) + \kappa(t) \sin^2\bigg(\frac{\theta_1 - \theta_2}{2}\bigg)
\end{equation}

\item[Usage]
Secondary publications and information retrieval purposes.
\item[Structure]
You may use the \texttt{description} environment to structure your abstract;
use the optional argument of the \verb+\item+ command to give the category of each item.
\end{description}
\end{abstract}

%\keywords{Suggested keywords}%Use showkeys class option if keyword
                              %display desired
\maketitle

%\tableofcontents

\section{\label{sec:level1}}

\end{document}
%
% ****** End of file apssamp.tex ******
 
